% Options for packages loaded elsewhere
% Options for packages loaded elsewhere
\PassOptionsToPackage{unicode}{hyperref}
\PassOptionsToPackage{hyphens}{url}
\PassOptionsToPackage{dvipsnames,svgnames,x11names}{xcolor}
%
\documentclass[
  british,
]{article}
\usepackage{xcolor}
\usepackage{amsmath,amssymb}
\setcounter{secnumdepth}{5}
\usepackage{iftex}
\ifPDFTeX
  \usepackage[T1]{fontenc}
  \usepackage[utf8]{inputenc}
  \usepackage{textcomp} % provide euro and other symbols
\else % if luatex or xetex
  \usepackage{unicode-math} % this also loads fontspec
  \defaultfontfeatures{Scale=MatchLowercase}
  \defaultfontfeatures[\rmfamily]{Ligatures=TeX,Scale=1}
\fi
\usepackage{lmodern}
\ifPDFTeX\else
  % xetex/luatex font selection
\fi
% Use upquote if available, for straight quotes in verbatim environments
\IfFileExists{upquote.sty}{\usepackage{upquote}}{}
\IfFileExists{microtype.sty}{% use microtype if available
  \usepackage[]{microtype}
  \UseMicrotypeSet[protrusion]{basicmath} % disable protrusion for tt fonts
}{}
\makeatletter
\@ifundefined{KOMAClassName}{% if non-KOMA class
  \IfFileExists{parskip.sty}{%
    \usepackage{parskip}
  }{% else
    \setlength{\parindent}{0pt}
    \setlength{\parskip}{6pt plus 2pt minus 1pt}}
}{% if KOMA class
  \KOMAoptions{parskip=half}}
\makeatother
% Make \paragraph and \subparagraph free-standing
\makeatletter
\ifx\paragraph\undefined\else
  \let\oldparagraph\paragraph
  \renewcommand{\paragraph}{
    \@ifstar
      \xxxParagraphStar
      \xxxParagraphNoStar
  }
  \newcommand{\xxxParagraphStar}[1]{\oldparagraph*{#1}\mbox{}}
  \newcommand{\xxxParagraphNoStar}[1]{\oldparagraph{#1}\mbox{}}
\fi
\ifx\subparagraph\undefined\else
  \let\oldsubparagraph\subparagraph
  \renewcommand{\subparagraph}{
    \@ifstar
      \xxxSubParagraphStar
      \xxxSubParagraphNoStar
  }
  \newcommand{\xxxSubParagraphStar}[1]{\oldsubparagraph*{#1}\mbox{}}
  \newcommand{\xxxSubParagraphNoStar}[1]{\oldsubparagraph{#1}\mbox{}}
\fi
\makeatother


\usepackage{longtable,booktabs,array}
\usepackage{calc} % for calculating minipage widths
% Correct order of tables after \paragraph or \subparagraph
\usepackage{etoolbox}
\makeatletter
\patchcmd\longtable{\par}{\if@noskipsec\mbox{}\fi\par}{}{}
\makeatother
% Allow footnotes in longtable head/foot
\IfFileExists{footnotehyper.sty}{\usepackage{footnotehyper}}{\usepackage{footnote}}
\makesavenoteenv{longtable}
\usepackage{graphicx}
\makeatletter
\newsavebox\pandoc@box
\newcommand*\pandocbounded[1]{% scales image to fit in text height/width
  \sbox\pandoc@box{#1}%
  \Gscale@div\@tempa{\textheight}{\dimexpr\ht\pandoc@box+\dp\pandoc@box\relax}%
  \Gscale@div\@tempb{\linewidth}{\wd\pandoc@box}%
  \ifdim\@tempb\p@<\@tempa\p@\let\@tempa\@tempb\fi% select the smaller of both
  \ifdim\@tempa\p@<\p@\scalebox{\@tempa}{\usebox\pandoc@box}%
  \else\usebox{\pandoc@box}%
  \fi%
}
% Set default figure placement to htbp
\def\fps@figure{htbp}
\makeatother


% definitions for citeproc citations
\NewDocumentCommand\citeproctext{}{}
\NewDocumentCommand\citeproc{mm}{%
  \begingroup\def\citeproctext{#2}\cite{#1}\endgroup}
\makeatletter
 % allow citations to break across lines
 \let\@cite@ofmt\@firstofone
 % avoid brackets around text for \cite:
 \def\@biblabel#1{}
 \def\@cite#1#2{{#1\if@tempswa , #2\fi}}
\makeatother
\newlength{\cslhangindent}
\setlength{\cslhangindent}{1.5em}
\newlength{\csllabelwidth}
\setlength{\csllabelwidth}{3em}
\newenvironment{CSLReferences}[2] % #1 hanging-indent, #2 entry-spacing
 {\begin{list}{}{%
  \setlength{\itemindent}{0pt}
  \setlength{\leftmargin}{0pt}
  \setlength{\parsep}{0pt}
  % turn on hanging indent if param 1 is 1
  \ifodd #1
   \setlength{\leftmargin}{\cslhangindent}
   \setlength{\itemindent}{-1\cslhangindent}
  \fi
  % set entry spacing
  \setlength{\itemsep}{#2\baselineskip}}}
 {\end{list}}
\usepackage{calc}
\newcommand{\CSLBlock}[1]{\hfill\break\parbox[t]{\linewidth}{\strut\ignorespaces#1\strut}}
\newcommand{\CSLLeftMargin}[1]{\parbox[t]{\csllabelwidth}{\strut#1\strut}}
\newcommand{\CSLRightInline}[1]{\parbox[t]{\linewidth - \csllabelwidth}{\strut#1\strut}}
\newcommand{\CSLIndent}[1]{\hspace{\cslhangindent}#1}

\ifLuaTeX
\usepackage[bidi=basic]{babel}
\else
\usepackage[bidi=default]{babel}
\fi
% get rid of language-specific shorthands (see #6817):
\let\LanguageShortHands\languageshorthands
\def\languageshorthands#1{}
\ifLuaTeX
  \usepackage[english]{selnolig} % disable illegal ligatures
\fi


\setlength{\emergencystretch}{3em} % prevent overfull lines

\providecommand{\tightlist}{%
  \setlength{\itemsep}{0pt}\setlength{\parskip}{0pt}}



 


% Get Math Fonts
\usepackage{fontspec}
\usepackage{unicode-math}
\setmathfont{Latin Modern Math}
\setmathfont[range={\mathscr,\mathbfscr,\mathbb}]{XITSMath-Regular}
[    Extension = .otf,
      BoldFont = XITSMath-Bold
]
\makeatletter
\@ifpackageloaded{caption}{}{\usepackage{caption}}
\AtBeginDocument{%
\ifdefined\contentsname
  \renewcommand*\contentsname{Table of contents}
\else
  \newcommand\contentsname{Table of contents}
\fi
\ifdefined\listfigurename
  \renewcommand*\listfigurename{List of Figures}
\else
  \newcommand\listfigurename{List of Figures}
\fi
\ifdefined\listtablename
  \renewcommand*\listtablename{List of Tables}
\else
  \newcommand\listtablename{List of Tables}
\fi
\ifdefined\figurename
  \renewcommand*\figurename{Figure}
\else
  \newcommand\figurename{Figure}
\fi
\ifdefined\tablename
  \renewcommand*\tablename{Table}
\else
  \newcommand\tablename{Table}
\fi
}
\@ifpackageloaded{float}{}{\usepackage{float}}
\floatstyle{ruled}
\@ifundefined{c@chapter}{\newfloat{codelisting}{h}{lop}}{\newfloat{codelisting}{h}{lop}[chapter]}
\floatname{codelisting}{Listing}
\newcommand*\listoflistings{\listof{codelisting}{List of Listings}}
\usepackage{amsthm}
\theoremstyle{definition}
\newtheorem{definition}{Definition}[section]
\theoremstyle{plain}
\newtheorem{proposition}{Proposition}[section]
\theoremstyle{remark}
\AtBeginDocument{\renewcommand*{\proofname}{Proof}}
\newtheorem*{remark}{Remark}
\newtheorem*{solution}{Solution}
\newtheorem{refremark}{Remark}[section]
\newtheorem{refsolution}{Solution}[section]
\makeatother
\makeatletter
\makeatother
\makeatletter
\@ifpackageloaded{caption}{}{\usepackage{caption}}
\@ifpackageloaded{subcaption}{}{\usepackage{subcaption}}
\makeatother

\newcommand{\GroupElement}{{\gamma}}
\newcommand{\Inverse}[1]{{#1}^{-1}}
\newcommand{\Folner}[1][\vphantom{}]{{\Phi}_{#1}}
\newcommand{\MonoidOperation}[2]{{#1}\cdot{#2}}
\newcommand{\CountingMeasure}{{\lambda}}
\newcommand{\Group}{{\Gamma}}
\newcommand{\MonoidElement}{{m}}
\newcommand{\Monoid}{{M}}
\newcommand{\GroupOperation}[2]{{#1}\cdot{#2}}

\usepackage{bookmark}
\IfFileExists{xurl.sty}{\usepackage{xurl}}{} % add URL line breaks if available
\urlstyle{same}
\hypersetup{
  pdftitle={Følner sequence},
  pdfauthor={Kai Prince},
  pdflang={en-GB},
  colorlinks=true,
  linkcolor={blue},
  filecolor={Maroon},
  citecolor={Blue},
  urlcolor={Blue},
  pdfcreator={LaTeX via pandoc}}


\title{Følner sequence}
\author{Kai Prince}
\date{2025-08-13}
\begin{document}
\maketitle

\renewcommand*\contentsname{Table of contents}
{
\hypersetup{linkcolor=}
\setcounter{tocdepth}{3}
\tableofcontents
}

\section{Følner Sequences}\label{fuxf8lner-sequences}

\begin{definition}[]\protect\hypertarget{def-rightFolner}{}\label{def-rightFolner}

We define a \emph{right-Følner sequence} in \(\Group\) as a sequence
\(\Folner =(\Folner[N])_{N\in\mathbb{N}}\) of finite subsets of
\(\Group\) satisfying
\[\lim_{N\rightarrow\infty}\frac{\CountingMeasure{\GroupOperation{(\GroupOperation{\Folner[N]}{\Inverse{\GroupElement}})}{\Folner[N]}}}{\CountingMeasure{\Folner[N]}}=1,\]
for all \(\GroupElement\in\Group\).

\end{definition}

\begin{definition}[]\protect\hypertarget{def-leftFolner}{}\label{def-leftFolner}

Similarly, we define a \emph{left-Følner sequence} in \(\Group\) as a
sequence \(\Folner =(\Folner[N])_{N\in\mathbb{N}}\) of finite subsets of
\(\Group\) satisfying
\[\lim_{N\rightarrow\infty}\frac{\CountingMeasure{(\Inverse{\GroupElement}\cdot\Folner[N])\cap\Folner[N]}}{\CountingMeasure{\Folner[N]}}=1,\]
for all \(\GroupElement\in\Group\).

\end{definition}

\begin{definition}[]\protect\hypertarget{def-Folner}{}\label{def-Folner}

We call a sequence in \(\Group\) a \emph{Følner sequence} if it is both
a left and right Følner sequence.

\end{definition}

\subsection{Alternative definitions for
Monoids}\label{alternative-definitions-for-monoids}

\begin{definition}[]\protect\hypertarget{def-rightFolnerMonoid}{}\label{def-rightFolnerMonoid}

Let \(\Monoid\) be a countably-infinite left-cancellative monoid with
discrete topology. We define a \emph{left-Følner sequence} in
\(\Monoid\) as a sequence of finite subsets
\(\Folner =(\Folner[N])_{N\in\mathbb{N}}\) satisfying
\[\lim_{N\rightarrow\infty}\frac{\CountingMeasure{\MonoidOperation{(\MonoidOperation{\MonoidElement}{\Folner[N]})}{\Folner[N]}}}{\CountingMeasure{\Folner[N]}}=1\]
for all \(g\in M\).

\end{definition}

\begin{definition}[]\protect\hypertarget{def-leftFolnerMonoid}{}\label{def-leftFolnerMonoid}

Similarly, for a countably-infinite right-cancellative monoid with
discrete topology \(\Monoid\), we define a \emph{right-Følner sequence}
in \(\Monoid\) as a sequence of finite subsets
\(\Folner =(\Folner[N])_{N\in\mathbb{N}}\) satisfying
\[\lim_{N\rightarrow\infty}\frac{\CountingMeasure{\MonoidOperation{(\MonoidOperation{\Folner[N]}{\MonoidElement})}{\Folner[N]}}}{\CountingMeasure{\Folner[N]}}=1\]
for all \(g\in M\).

\end{definition}

\subsection{Equivalent definitions using Set
Differences}\label{equivalent-definitions-using-set-differences}

Equivalent definitions can be constructed by using set differences
instead of intersections.

For example, the equivalent definition of a left-Følner sequence,
\(\Folner\), in \(\Monoid\) requires
\[\lim_{N\rightarrow\infty}\frac{\CountingMeasure{(\MonoidOperation{\Folner[N]}{\MonoidElement})\triangle\Folner[N]}}{\CountingMeasure{\Folner[N]}}=0, \]
to be satisfied for all \(\MonoidElement\in\Monoid\).

This alternative definition will be useful when looking at proving some
of the properties of density.

\section{Tempered Følner Sequences}\label{tempered-fuxf8lner-sequences}

\begin{definition}[Lindenstrauss
\citeproc{ref-Lindenstrauss2001PETAmenable}{(2001)}, Definition
1.1]\protect\hypertarget{def-tempered}{}\label{def-tempered}

A sequence of sets \(\Folner=(\Folner[N])_{N\in\mathbb{N}}\) will be
said to be \emph{tempered} if, for some \(b>0\) and all
\(n\in\mathbb{N}\),

\begin{equation}\phantomsection\label{eq-ShulmanCondition}{
\CountingMeasure{\bigcup_{1\leq k<N}\Folner[k]^{-1}\Folner[N]}\leq b\CountingMeasure{\Folner[N]}.
}\end{equation}

is referred to as the \emph{Shulman condition}.

\end{definition}

\begin{proposition}[Lindenstrauss
\citeproc{ref-Lindenstrauss2001PETAmenable}{(2001)}, Proposition
1.4]\protect\hypertarget{prp-temperedFolner}{}\label{prp-temperedFolner}

~

\begin{enumerate}
\def\labelenumi{\arabic{enumi}.}
\tightlist
\item
  Every Følner sequence \(\Folner=(\Folner[N])_{N\in\mathbb{N}}\) has a
  tempered subsequence.
\item
  Every amenable group has a tempered Følner sequence.
\end{enumerate}

\end{proposition}

\phantomsection\label{refs}
\begin{CSLReferences}{1}{0}
\bibitem[\citeproctext]{ref-Lindenstrauss2001PETAmenable}
Lindenstrauss, E. (2001). \textquotesingle Pointwise theorems for
amenable groups\textquotesingle, \emph{\emph{Inventiones mathematicae},}
146 (2), pp. 259--295.
\href{https://doi.org/10.1007/s002220100162}{https://doi.org/10.1007/s002220100162.}

\end{CSLReferences}




\end{document}
