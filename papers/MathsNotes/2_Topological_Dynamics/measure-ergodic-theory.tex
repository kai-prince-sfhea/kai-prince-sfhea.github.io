% Options for packages loaded elsewhere
% Options for packages loaded elsewhere
\PassOptionsToPackage{unicode}{hyperref}
\PassOptionsToPackage{hyphens}{url}
\PassOptionsToPackage{dvipsnames,svgnames,x11names}{xcolor}
%
\documentclass[
  british,
]{article}
\usepackage{xcolor}
\usepackage{amsmath,amssymb}
\setcounter{secnumdepth}{5}
\usepackage{iftex}
\ifPDFTeX
  \usepackage[T1]{fontenc}
  \usepackage[utf8]{inputenc}
  \usepackage{textcomp} % provide euro and other symbols
\else % if luatex or xetex
  \usepackage{unicode-math} % this also loads fontspec
  \defaultfontfeatures{Scale=MatchLowercase}
  \defaultfontfeatures[\rmfamily]{Ligatures=TeX,Scale=1}
\fi
\usepackage{lmodern}
\ifPDFTeX\else
  % xetex/luatex font selection
\fi
% Use upquote if available, for straight quotes in verbatim environments
\IfFileExists{upquote.sty}{\usepackage{upquote}}{}
\IfFileExists{microtype.sty}{% use microtype if available
  \usepackage[]{microtype}
  \UseMicrotypeSet[protrusion]{basicmath} % disable protrusion for tt fonts
}{}
\makeatletter
\@ifundefined{KOMAClassName}{% if non-KOMA class
  \IfFileExists{parskip.sty}{%
    \usepackage{parskip}
  }{% else
    \setlength{\parindent}{0pt}
    \setlength{\parskip}{6pt plus 2pt minus 1pt}}
}{% if KOMA class
  \KOMAoptions{parskip=half}}
\makeatother
% Make \paragraph and \subparagraph free-standing
\makeatletter
\ifx\paragraph\undefined\else
  \let\oldparagraph\paragraph
  \renewcommand{\paragraph}{
    \@ifstar
      \xxxParagraphStar
      \xxxParagraphNoStar
  }
  \newcommand{\xxxParagraphStar}[1]{\oldparagraph*{#1}\mbox{}}
  \newcommand{\xxxParagraphNoStar}[1]{\oldparagraph{#1}\mbox{}}
\fi
\ifx\subparagraph\undefined\else
  \let\oldsubparagraph\subparagraph
  \renewcommand{\subparagraph}{
    \@ifstar
      \xxxSubParagraphStar
      \xxxSubParagraphNoStar
  }
  \newcommand{\xxxSubParagraphStar}[1]{\oldsubparagraph*{#1}\mbox{}}
  \newcommand{\xxxSubParagraphNoStar}[1]{\oldsubparagraph{#1}\mbox{}}
\fi
\makeatother


\usepackage{longtable,booktabs,array}
\usepackage{calc} % for calculating minipage widths
% Correct order of tables after \paragraph or \subparagraph
\usepackage{etoolbox}
\makeatletter
\patchcmd\longtable{\par}{\if@noskipsec\mbox{}\fi\par}{}{}
\makeatother
% Allow footnotes in longtable head/foot
\IfFileExists{footnotehyper.sty}{\usepackage{footnotehyper}}{\usepackage{footnote}}
\makesavenoteenv{longtable}
\usepackage{graphicx}
\makeatletter
\newsavebox\pandoc@box
\newcommand*\pandocbounded[1]{% scales image to fit in text height/width
  \sbox\pandoc@box{#1}%
  \Gscale@div\@tempa{\textheight}{\dimexpr\ht\pandoc@box+\dp\pandoc@box\relax}%
  \Gscale@div\@tempb{\linewidth}{\wd\pandoc@box}%
  \ifdim\@tempb\p@<\@tempa\p@\let\@tempa\@tempb\fi% select the smaller of both
  \ifdim\@tempa\p@<\p@\scalebox{\@tempa}{\usebox\pandoc@box}%
  \else\usebox{\pandoc@box}%
  \fi%
}
% Set default figure placement to htbp
\def\fps@figure{htbp}
\makeatother


% definitions for citeproc citations
\NewDocumentCommand\citeproctext{}{}
\NewDocumentCommand\citeproc{mm}{%
  \begingroup\def\citeproctext{#2}\cite{#1}\endgroup}
\makeatletter
 % allow citations to break across lines
 \let\@cite@ofmt\@firstofone
 % avoid brackets around text for \cite:
 \def\@biblabel#1{}
 \def\@cite#1#2{{#1\if@tempswa , #2\fi}}
\makeatother
\newlength{\cslhangindent}
\setlength{\cslhangindent}{1.5em}
\newlength{\csllabelwidth}
\setlength{\csllabelwidth}{3em}
\newenvironment{CSLReferences}[2] % #1 hanging-indent, #2 entry-spacing
 {\begin{list}{}{%
  \setlength{\itemindent}{0pt}
  \setlength{\leftmargin}{0pt}
  \setlength{\parsep}{0pt}
  % turn on hanging indent if param 1 is 1
  \ifodd #1
   \setlength{\leftmargin}{\cslhangindent}
   \setlength{\itemindent}{-1\cslhangindent}
  \fi
  % set entry spacing
  \setlength{\itemsep}{#2\baselineskip}}}
 {\end{list}}
\usepackage{calc}
\newcommand{\CSLBlock}[1]{\hfill\break\parbox[t]{\linewidth}{\strut\ignorespaces#1\strut}}
\newcommand{\CSLLeftMargin}[1]{\parbox[t]{\csllabelwidth}{\strut#1\strut}}
\newcommand{\CSLRightInline}[1]{\parbox[t]{\linewidth - \csllabelwidth}{\strut#1\strut}}
\newcommand{\CSLIndent}[1]{\hspace{\cslhangindent}#1}

\ifLuaTeX
\usepackage[bidi=basic]{babel}
\else
\usepackage[bidi=default]{babel}
\fi
% get rid of language-specific shorthands (see #6817):
\let\LanguageShortHands\languageshorthands
\def\languageshorthands#1{}
\ifLuaTeX
  \usepackage[english]{selnolig} % disable illegal ligatures
\fi


\setlength{\emergencystretch}{3em} % prevent overfull lines

\providecommand{\tightlist}{%
  \setlength{\itemsep}{0pt}\setlength{\parskip}{0pt}}



 


% Get Math Fonts
\usepackage{fontspec}
\usepackage{unicode-math}
\setmathfont{Latin Modern Math}
\setmathfont[range={\mathscr,\mathbfscr,\mathbb}]{XITSMath-Regular}
[    Extension = .otf,
      BoldFont = XITSMath-Bold
]
\makeatletter
\@ifpackageloaded{caption}{}{\usepackage{caption}}
\AtBeginDocument{%
\ifdefined\contentsname
  \renewcommand*\contentsname{Table of contents}
\else
  \newcommand\contentsname{Table of contents}
\fi
\ifdefined\listfigurename
  \renewcommand*\listfigurename{List of Figures}
\else
  \newcommand\listfigurename{List of Figures}
\fi
\ifdefined\listtablename
  \renewcommand*\listtablename{List of Tables}
\else
  \newcommand\listtablename{List of Tables}
\fi
\ifdefined\figurename
  \renewcommand*\figurename{Figure}
\else
  \newcommand\figurename{Figure}
\fi
\ifdefined\tablename
  \renewcommand*\tablename{Table}
\else
  \newcommand\tablename{Table}
\fi
}
\@ifpackageloaded{float}{}{\usepackage{float}}
\floatstyle{ruled}
\@ifundefined{c@chapter}{\newfloat{codelisting}{h}{lop}}{\newfloat{codelisting}{h}{lop}[chapter]}
\floatname{codelisting}{Listing}
\newcommand*\listoflistings{\listof{codelisting}{List of Listings}}
\usepackage{amsthm}
\theoremstyle{definition}
\newtheorem{definition}{Definition}[section]
\theoremstyle{remark}
\AtBeginDocument{\renewcommand*{\proofname}{Proof}}
\newtheorem*{remark}{Remark}
\newtheorem*{solution}{Solution}
\newtheorem{refremark}{Remark}[section]
\newtheorem{refsolution}{Solution}[section]
\makeatother
\makeatletter
\makeatother
\makeatletter
\@ifpackageloaded{caption}{}{\usepackage{caption}}
\@ifpackageloaded{subcaption}{}{\usepackage{subcaption}}
\makeatother

\newcommand{\GroupElement}{{\gamma}}
\newcommand{\GroupOperation}[2]{{#1}\cdot{#2}}
\newcommand{\Topology}{{\tau}}
\newcommand{\SigmaAlgebraGenerator}[1][\Topology]{{\sigma({#1})}}
\newcommand{\Group}{{\Gamma}}
\newcommand{\Measure}{{\mu}}
\newcommand{\SigmaAlgebra}[1]{{\mathscr{#1}}}

\usepackage{bookmark}
\IfFileExists{xurl.sty}{\usepackage{xurl}}{} % add URL line breaks if available
\urlstyle{same}
\hypersetup{
  pdftitle={Measure Theory and Ergodic Theory},
  pdfauthor={Kai Prince},
  pdflang={en-GB},
  colorlinks=true,
  linkcolor={blue},
  filecolor={Maroon},
  citecolor={Blue},
  urlcolor={Blue},
  pdfcreator={LaTeX via pandoc}}


\title{Measure Theory and Ergodic Theory}
\author{Kai Prince}
\date{2025-08-14}
\begin{document}
\maketitle

\renewcommand*\contentsname{Table of contents}
{
\hypersetup{linkcolor=}
\setcounter{tocdepth}{3}
\tableofcontents
}

\section*{Backlinks}\label{sec-Backlinks}
\addcontentsline{toc}{section}{Backlinks}

\begin{itemize}
\tightlist
\item
  \href{../../MathsNotes/0_Fundamental/group.html}{Group}
\end{itemize}

The foundational knowledge relating to Measure \& Ergodic Theory, that
has not been covered elsewhere, was provided by Robertson
\citeproc{ref-Robertson2023MeasureErgodicTheory}{(2023)}.

\begin{definition}[Gleason
\citeproc{ref-Gleason2010HaarMeasure}{(2010)}]\protect\hypertarget{def-TopoMeasureSpace}{}\label{def-TopoMeasureSpace}

A \emph{topological measure space} \((X,\SigmaAlgebra{B},\Measure)\) is
a topological space \((X,\Topology)\) such that \(\SigmaAlgebra{B}\) is
generated by the open sets defined by the topology \(\Topology\), i.e.,
\(\SigmaAlgebra{B}=\text{Borel}(X)=\SigmaAlgebraGenerator\), and
\(\Measure\) is a measure on this space.

A \emph{Borel measure} is the measure \(\Measure\) on a topological
measure space \((X,\SigmaAlgebra{B},\Measure)\) where \((X,\Topology)\)
is Hausdorff.

A \emph{regular (Borel) measure} is a measure on a Borel measure space
\((X,\SigmaAlgebra{B},\Measure)\) such that the following hold:

\begin{enumerate}
    \item *Finite Compact Measure*: For any compact subset $K\subseteq X$, then $\Measure(K)<\infty$.
    \item *Outer Regularity*: For any $B\in\SigmaAlgebra{B}$, then
    $$\Measure(B)=\inf\{\Measure(C)\mid B\subseteq C, C\text{ is open} \}. $$
    \item *Inner Regularity*: For any $U\in\Topology$, or, in other words, any open subset $U\subseteq X$, then
    $$\Measure(U)=\sup\{\Measure(K)\mid K\subseteq U, K\text{ compact} \}. $$
\end{enumerate}

A \emph{left-Haar measure} {[}or \emph{right-Haar measure}{]} on a
topological group \((\Group,\Topology)\) is a non-zero regular Borel
measure \(\Measure\) on \(\Group\) such that
\(\Measure(\GroupOperation{\GroupElement}{B})=\Measure(B)\) {[}or
\(\Measure(\GroupOperation{B}{\GroupElement})=\Measure(B)\){]} for all
\(\GroupElement\in\Group\) and \(B\in\sigma(\Topology)\).

\end{definition}

\section*{Outlinks}\label{sec-Outlinks}
\addcontentsline{toc}{section}{Outlinks}

\begin{itemize}
\tightlist
\item
  \href{../../MathsNotes/1_Amenable_and_Density/density.html}{Density}
\item
  \href{../../MathsNotes/1_Amenable_and_Density/folner.html}{Følner sequence}
\item
  \href{../../MathsNotes/2_Topological_Dynamics/actions.html}{Actions}
\item
  \href{../../MathsNotes/2_Topological_Dynamics/factor-maps.html}{Factor Maps}
\item
  \href{../../MathsNotes/2_Topological_Dynamics/furstenberg-correspondence.html}{Furstenberg's Correspondence Principle}
\item
  \href{../../GeneralisedErdosConjecture/ErdosConjecture.html}{A Short Proof of a Generalised Conjecture of Erdős for Amenable Groups}
\item
  \href{../../MathsNotes/2_Topological_Dynamics/recurrence-and-Ergodic-theorem.html}{Recurrence and Ergodic Theorems}
\end{itemize}

\phantomsection\label{refs}
\begin{CSLReferences}{1}{0}
\bibitem[\citeproctext]{ref-Robertson2023MeasureErgodicTheory}
Robertson, D. (2023). \textquotesingle{}\emph{MATH41021/61021 measure
and ergodic theory}\textquotesingle, Available at:
\url{https://personalpages.manchester.ac.uk/staff/donald.robertson/teaching/23-24/41021}
(Accessed: 22 January 2024).

\bibitem[\citeproctext]{ref-Gleason2010HaarMeasure}
Gleason, J. (2010). \textquotesingle{}\emph{Existence and uniqueness of
haar measure}\textquotesingle,.

\end{CSLReferences}




\end{document}
