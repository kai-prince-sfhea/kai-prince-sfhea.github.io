% Options for packages loaded elsewhere
% Options for packages loaded elsewhere
\PassOptionsToPackage{unicode}{hyperref}
\PassOptionsToPackage{hyphens}{url}
\PassOptionsToPackage{dvipsnames,svgnames,x11names}{xcolor}
%
\documentclass[
  british,
]{article}
\usepackage{xcolor}
\usepackage{amsmath,amssymb}
\setcounter{secnumdepth}{5}
\usepackage{iftex}
\ifPDFTeX
  \usepackage[T1]{fontenc}
  \usepackage[utf8]{inputenc}
  \usepackage{textcomp} % provide euro and other symbols
\else % if luatex or xetex
  \usepackage{unicode-math} % this also loads fontspec
  \defaultfontfeatures{Scale=MatchLowercase}
  \defaultfontfeatures[\rmfamily]{Ligatures=TeX,Scale=1}
\fi
\usepackage{lmodern}
\ifPDFTeX\else
  % xetex/luatex font selection
\fi
% Use upquote if available, for straight quotes in verbatim environments
\IfFileExists{upquote.sty}{\usepackage{upquote}}{}
\IfFileExists{microtype.sty}{% use microtype if available
  \usepackage[]{microtype}
  \UseMicrotypeSet[protrusion]{basicmath} % disable protrusion for tt fonts
}{}
\makeatletter
\@ifundefined{KOMAClassName}{% if non-KOMA class
  \IfFileExists{parskip.sty}{%
    \usepackage{parskip}
  }{% else
    \setlength{\parindent}{0pt}
    \setlength{\parskip}{6pt plus 2pt minus 1pt}}
}{% if KOMA class
  \KOMAoptions{parskip=half}}
\makeatother
% Make \paragraph and \subparagraph free-standing
\makeatletter
\ifx\paragraph\undefined\else
  \let\oldparagraph\paragraph
  \renewcommand{\paragraph}{
    \@ifstar
      \xxxParagraphStar
      \xxxParagraphNoStar
  }
  \newcommand{\xxxParagraphStar}[1]{\oldparagraph*{#1}\mbox{}}
  \newcommand{\xxxParagraphNoStar}[1]{\oldparagraph{#1}\mbox{}}
\fi
\ifx\subparagraph\undefined\else
  \let\oldsubparagraph\subparagraph
  \renewcommand{\subparagraph}{
    \@ifstar
      \xxxSubParagraphStar
      \xxxSubParagraphNoStar
  }
  \newcommand{\xxxSubParagraphStar}[1]{\oldsubparagraph*{#1}\mbox{}}
  \newcommand{\xxxSubParagraphNoStar}[1]{\oldsubparagraph{#1}\mbox{}}
\fi
\makeatother


\usepackage{longtable,booktabs,array}
\usepackage{calc} % for calculating minipage widths
% Correct order of tables after \paragraph or \subparagraph
\usepackage{etoolbox}
\makeatletter
\patchcmd\longtable{\par}{\if@noskipsec\mbox{}\fi\par}{}{}
\makeatother
% Allow footnotes in longtable head/foot
\IfFileExists{footnotehyper.sty}{\usepackage{footnotehyper}}{\usepackage{footnote}}
\makesavenoteenv{longtable}
\usepackage{graphicx}
\makeatletter
\newsavebox\pandoc@box
\newcommand*\pandocbounded[1]{% scales image to fit in text height/width
  \sbox\pandoc@box{#1}%
  \Gscale@div\@tempa{\textheight}{\dimexpr\ht\pandoc@box+\dp\pandoc@box\relax}%
  \Gscale@div\@tempb{\linewidth}{\wd\pandoc@box}%
  \ifdim\@tempb\p@<\@tempa\p@\let\@tempa\@tempb\fi% select the smaller of both
  \ifdim\@tempa\p@<\p@\scalebox{\@tempa}{\usebox\pandoc@box}%
  \else\usebox{\pandoc@box}%
  \fi%
}
% Set default figure placement to htbp
\def\fps@figure{htbp}
\makeatother


% definitions for citeproc citations
\NewDocumentCommand\citeproctext{}{}
\NewDocumentCommand\citeproc{mm}{%
  \begingroup\def\citeproctext{#2}\cite{#1}\endgroup}
\makeatletter
 % allow citations to break across lines
 \let\@cite@ofmt\@firstofone
 % avoid brackets around text for \cite:
 \def\@biblabel#1{}
 \def\@cite#1#2{{#1\if@tempswa , #2\fi}}
\makeatother
\newlength{\cslhangindent}
\setlength{\cslhangindent}{1.5em}
\newlength{\csllabelwidth}
\setlength{\csllabelwidth}{3em}
\newenvironment{CSLReferences}[2] % #1 hanging-indent, #2 entry-spacing
 {\begin{list}{}{%
  \setlength{\itemindent}{0pt}
  \setlength{\leftmargin}{0pt}
  \setlength{\parsep}{0pt}
  % turn on hanging indent if param 1 is 1
  \ifodd #1
   \setlength{\leftmargin}{\cslhangindent}
   \setlength{\itemindent}{-1\cslhangindent}
  \fi
  % set entry spacing
  \setlength{\itemsep}{#2\baselineskip}}}
 {\end{list}}
\usepackage{calc}
\newcommand{\CSLBlock}[1]{\hfill\break\parbox[t]{\linewidth}{\strut\ignorespaces#1\strut}}
\newcommand{\CSLLeftMargin}[1]{\parbox[t]{\csllabelwidth}{\strut#1\strut}}
\newcommand{\CSLRightInline}[1]{\parbox[t]{\linewidth - \csllabelwidth}{\strut#1\strut}}
\newcommand{\CSLIndent}[1]{\hspace{\cslhangindent}#1}

\ifLuaTeX
\usepackage[bidi=basic]{babel}
\else
\usepackage[bidi=default]{babel}
\fi
% get rid of language-specific shorthands (see #6817):
\let\LanguageShortHands\languageshorthands
\def\languageshorthands#1{}
\ifLuaTeX
  \usepackage[english]{selnolig} % disable illegal ligatures
\fi


\setlength{\emergencystretch}{3em} % prevent overfull lines

\providecommand{\tightlist}{%
  \setlength{\itemsep}{0pt}\setlength{\parskip}{0pt}}



 


% Get Math Fonts
\usepackage{fontspec}
\usepackage{unicode-math}
\setmathfont{Latin Modern Math}
\setmathfont[range={\mathscr,\mathbfscr,\mathbb}]{XITSMath-Regular}
[    Extension = .otf,
      BoldFont = XITSMath-Bold
]
\makeatletter
\@ifpackageloaded{caption}{}{\usepackage{caption}}
\AtBeginDocument{%
\ifdefined\contentsname
  \renewcommand*\contentsname{Table of contents}
\else
  \newcommand\contentsname{Table of contents}
\fi
\ifdefined\listfigurename
  \renewcommand*\listfigurename{List of Figures}
\else
  \newcommand\listfigurename{List of Figures}
\fi
\ifdefined\listtablename
  \renewcommand*\listtablename{List of Tables}
\else
  \newcommand\listtablename{List of Tables}
\fi
\ifdefined\figurename
  \renewcommand*\figurename{Figure}
\else
  \newcommand\figurename{Figure}
\fi
\ifdefined\tablename
  \renewcommand*\tablename{Table}
\else
  \newcommand\tablename{Table}
\fi
}
\@ifpackageloaded{float}{}{\usepackage{float}}
\floatstyle{ruled}
\@ifundefined{c@chapter}{\newfloat{codelisting}{h}{lop}}{\newfloat{codelisting}{h}{lop}[chapter]}
\floatname{codelisting}{Listing}
\newcommand*\listoflistings{\listof{codelisting}{List of Listings}}
\usepackage{amsthm}
\theoremstyle{plain}
\newtheorem{proposition}{Proposition}[section]
\theoremstyle{definition}
\newtheorem{definition}{Definition}[section]
\theoremstyle{remark}
\AtBeginDocument{\renewcommand*{\proofname}{Proof}}
\newtheorem*{remark}{Remark}
\newtheorem*{solution}{Solution}
\newtheorem{refremark}{Remark}[section]
\newtheorem{refsolution}{Solution}[section]
\makeatother
\makeatletter
\makeatother
\makeatletter
\@ifpackageloaded{caption}{}{\usepackage{caption}}
\@ifpackageloaded{subcaption}{}{\usepackage{subcaption}}
\makeatother

\newcommand{\Folner}[1][\vphantom{}]{{\Phi}_{#1}}
\newcommand{\AmenableGroup}{{\Gamma}}
\newcommand{\ProjectionMap}[1][\vphantom{}]{{\pi}_{#1}}
\newcommand{\KroneckerMeasure}{{m_Z}}
\newcommand{\Measure}{{\mu}}
\newcommand{\KroneckerSpace}{{Z}}
\newcommand{\KroneckerSpaceElement}{{z}}
\newcommand{\KroneckerAction}{{H}}
\newcommand{\KroneckerFactor}{{(\KroneckerSpace,\KroneckerMeasure,\KroneckerAction)}}

\usepackage{bookmark}
\IfFileExists{xurl.sty}{\usepackage{xurl}}{} % add URL line breaks if available
\urlstyle{same}
\hypersetup{
  pdftitle={Factor Maps},
  pdfauthor={Kai Prince},
  pdflang={en-GB},
  colorlinks=true,
  linkcolor={blue},
  filecolor={Maroon},
  citecolor={Blue},
  urlcolor={Blue},
  pdfcreator={LaTeX via pandoc}}


\title{Factor Maps}
\author{Kai Prince}
\date{2025-08-14}
\begin{document}
\maketitle

\renewcommand*\contentsname{Table of contents}
{
\hypersetup{linkcolor=}
\setcounter{tocdepth}{3}
\tableofcontents
}

\section*{Backlinks}\label{sec-Backlinks}
\addcontentsline{toc}{section}{Backlinks}

\begin{itemize}
\tightlist
\item
  \href{../../MathsNotes/0_Fundamental/group.html}{Group}
\item
  \href{../../MathsNotes/2_Topological_Dynamics/measure-ergodic-theory.html}{Measure Theory and Ergodic Theory}
\item
  \href{../../MathsNotes/1_Amenable_and_Density/folner.html}{Følner sequence}
\item
  \href{../../MathsNotes/1_Amenable_and_Density/amenable.html}{Amenable}
\item
  \href{../../MathsNotes/2_Topological_Dynamics/actions.html}{Actions}
\end{itemize}

\begin{definition}[Kra et al. \citeproc{ref-kra2022infinite}{(2022)},
Definition
2.1]\protect\hypertarget{def-MeasurableFactor}{}\label{def-MeasurableFactor}

For a system \((X,\Measure,T)\) we say that the system \((Y,\
u,S)\) is a \emph{measurable factor} of \((X,\Measure,T)\) if there is a
measurable map \(\ProjectionMap:X\rightarrow Y\), the \emph{measurable
factor map}, such that \(\ProjectionMap(\Measure)=\
u\) and \((S\circ\ProjectionMap)(x)=(\ProjectionMap\circ T)(x)\) for
\(\Measure\)-almost every \(x\in X\).

\end{definition}

\begin{proposition}[cf. Host \citeproc{ref-host2019short}{(2019)},
Proposition
5]\protect\hypertarget{prp-KroneckerGeneric}{}\label{prp-KroneckerGeneric}

Let \((X,\AmenableGroup)\) be a topological dynamical system where
\(\AmenableGroup\) is an amenable group, \(x_0\in X\), and \(\Measure\)
be an ergodic invariant probability measure supported on the closed
orbit of \(x_0\) under the action of \(\AmenableGroup\).

Let \(\KroneckerFactor\) be the Kronecker factor of
\((X,\Measure,\AmenableGroup)\), with factor map
\(\ProjectionMap:X\rightarrow\KroneckerSpace\).

Let \(X\times\KroneckerSpace\) be endowed with the group action of
\(\AmenableGroup\times\KroneckerAction\). Let \(\tilde\Measure\) be the
measure on \(X\times\KroneckerSpace\) and image of \(\Measure\) under
the map \(X\rightarrow X\times\KroneckerSpace\) where
\(x\mapsto(x,\ProjectionMap(x))\).

Then there exists a Følner sequence \(\tilde{\Folner}\) and a point
\(\KroneckerSpaceElement_0\in\KroneckerSpace\) such that
\((x_0,\KroneckerSpaceElement_0)\) is generic for \(\tilde{\Measure}\)
along \(\tilde{\Folner}\).

\end{proposition}

\section*{Outlinks}\label{sec-Outlinks}
\addcontentsline{toc}{section}{Outlinks}

\begin{itemize}
\tightlist
\item
  \href{../../GeneralisedErdosConjecture/ErdosConjecture.html}{A Short Proof of a Generalised Conjecture of Erdős for Amenable Groups}
\end{itemize}

\phantomsection\label{refs}
\begin{CSLReferences}{1}{0}
\bibitem[\citeproctext]{ref-kra2022infinite}
Kra, B., et al. (2022). \textquotesingle{}\emph{Infinite sumsets in sets
with positive density}\textquotesingle, Available at:
\url{https://arxiv.org/abs/2206.01786}.

\bibitem[\citeproctext]{ref-host2019short}
Host, B. (2019). \textquotesingle{}\emph{A short proof of a conjecture
of erdös proved by moreira, richter and robertson}\textquotesingle,
Available at: \url{https://arxiv.org/abs/1904.09952}.

\end{CSLReferences}




\end{document}
