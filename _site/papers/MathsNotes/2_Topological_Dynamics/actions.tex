% Options for packages loaded elsewhere
% Options for packages loaded elsewhere
\PassOptionsToPackage{unicode}{hyperref}
\PassOptionsToPackage{hyphens}{url}
\PassOptionsToPackage{dvipsnames,svgnames,x11names}{xcolor}
%
\documentclass[
  british,
]{article}
\usepackage{xcolor}
\usepackage{amsmath,amssymb}
\setcounter{secnumdepth}{5}
\usepackage{iftex}
\ifPDFTeX
  \usepackage[T1]{fontenc}
  \usepackage[utf8]{inputenc}
  \usepackage{textcomp} % provide euro and other symbols
\else % if luatex or xetex
  \usepackage{unicode-math} % this also loads fontspec
  \defaultfontfeatures{Scale=MatchLowercase}
  \defaultfontfeatures[\rmfamily]{Ligatures=TeX,Scale=1}
\fi
\usepackage{lmodern}
\ifPDFTeX\else
  % xetex/luatex font selection
\fi
% Use upquote if available, for straight quotes in verbatim environments
\IfFileExists{upquote.sty}{\usepackage{upquote}}{}
\IfFileExists{microtype.sty}{% use microtype if available
  \usepackage[]{microtype}
  \UseMicrotypeSet[protrusion]{basicmath} % disable protrusion for tt fonts
}{}
\makeatletter
\@ifundefined{KOMAClassName}{% if non-KOMA class
  \IfFileExists{parskip.sty}{%
    \usepackage{parskip}
  }{% else
    \setlength{\parindent}{0pt}
    \setlength{\parskip}{6pt plus 2pt minus 1pt}}
}{% if KOMA class
  \KOMAoptions{parskip=half}}
\makeatother
% Make \paragraph and \subparagraph free-standing
\makeatletter
\ifx\paragraph\undefined\else
  \let\oldparagraph\paragraph
  \renewcommand{\paragraph}{
    \@ifstar
      \xxxParagraphStar
      \xxxParagraphNoStar
  }
  \newcommand{\xxxParagraphStar}[1]{\oldparagraph*{#1}\mbox{}}
  \newcommand{\xxxParagraphNoStar}[1]{\oldparagraph{#1}\mbox{}}
\fi
\ifx\subparagraph\undefined\else
  \let\oldsubparagraph\subparagraph
  \renewcommand{\subparagraph}{
    \@ifstar
      \xxxSubParagraphStar
      \xxxSubParagraphNoStar
  }
  \newcommand{\xxxSubParagraphStar}[1]{\oldsubparagraph*{#1}\mbox{}}
  \newcommand{\xxxSubParagraphNoStar}[1]{\oldsubparagraph{#1}\mbox{}}
\fi
\makeatother


\usepackage{longtable,booktabs,array}
\usepackage{calc} % for calculating minipage widths
% Correct order of tables after \paragraph or \subparagraph
\usepackage{etoolbox}
\makeatletter
\patchcmd\longtable{\par}{\if@noskipsec\mbox{}\fi\par}{}{}
\makeatother
% Allow footnotes in longtable head/foot
\IfFileExists{footnotehyper.sty}{\usepackage{footnotehyper}}{\usepackage{footnote}}
\makesavenoteenv{longtable}
\usepackage{graphicx}
\makeatletter
\newsavebox\pandoc@box
\newcommand*\pandocbounded[1]{% scales image to fit in text height/width
  \sbox\pandoc@box{#1}%
  \Gscale@div\@tempa{\textheight}{\dimexpr\ht\pandoc@box+\dp\pandoc@box\relax}%
  \Gscale@div\@tempb{\linewidth}{\wd\pandoc@box}%
  \ifdim\@tempb\p@<\@tempa\p@\let\@tempa\@tempb\fi% select the smaller of both
  \ifdim\@tempa\p@<\p@\scalebox{\@tempa}{\usebox\pandoc@box}%
  \else\usebox{\pandoc@box}%
  \fi%
}
% Set default figure placement to htbp
\def\fps@figure{htbp}
\makeatother


% definitions for citeproc citations
\NewDocumentCommand\citeproctext{}{}
\NewDocumentCommand\citeproc{mm}{%
  \begingroup\def\citeproctext{#2}\cite{#1}\endgroup}
\makeatletter
 % allow citations to break across lines
 \let\@cite@ofmt\@firstofone
 % avoid brackets around text for \cite:
 \def\@biblabel#1{}
 \def\@cite#1#2{{#1\if@tempswa , #2\fi}}
\makeatother
\newlength{\cslhangindent}
\setlength{\cslhangindent}{1.5em}
\newlength{\csllabelwidth}
\setlength{\csllabelwidth}{3em}
\newenvironment{CSLReferences}[2] % #1 hanging-indent, #2 entry-spacing
 {\begin{list}{}{%
  \setlength{\itemindent}{0pt}
  \setlength{\leftmargin}{0pt}
  \setlength{\parsep}{0pt}
  % turn on hanging indent if param 1 is 1
  \ifodd #1
   \setlength{\leftmargin}{\cslhangindent}
   \setlength{\itemindent}{-1\cslhangindent}
  \fi
  % set entry spacing
  \setlength{\itemsep}{#2\baselineskip}}}
 {\end{list}}
\usepackage{calc}
\newcommand{\CSLBlock}[1]{\hfill\break\parbox[t]{\linewidth}{\strut\ignorespaces#1\strut}}
\newcommand{\CSLLeftMargin}[1]{\parbox[t]{\csllabelwidth}{\strut#1\strut}}
\newcommand{\CSLRightInline}[1]{\parbox[t]{\linewidth - \csllabelwidth}{\strut#1\strut}}
\newcommand{\CSLIndent}[1]{\hspace{\cslhangindent}#1}

\ifLuaTeX
\usepackage[bidi=basic]{babel}
\else
\usepackage[bidi=default]{babel}
\fi
% get rid of language-specific shorthands (see #6817):
\let\LanguageShortHands\languageshorthands
\def\languageshorthands#1{}
\ifLuaTeX
  \usepackage[english]{selnolig} % disable illegal ligatures
\fi


\setlength{\emergencystretch}{3em} % prevent overfull lines

\providecommand{\tightlist}{%
  \setlength{\itemsep}{0pt}\setlength{\parskip}{0pt}}



 


% Get Math Fonts
\usepackage{fontspec}
\usepackage{unicode-math}
\setmathfont{Latin Modern Math}
\setmathfont[range={\mathscr,\mathbfscr,\mathbb}]{XITSMath-Regular}
[    Extension = .otf,
      BoldFont = XITSMath-Bold
]
\makeatletter
\@ifpackageloaded{caption}{}{\usepackage{caption}}
\AtBeginDocument{%
\ifdefined\contentsname
  \renewcommand*\contentsname{Table of contents}
\else
  \newcommand\contentsname{Table of contents}
\fi
\ifdefined\listfigurename
  \renewcommand*\listfigurename{List of Figures}
\else
  \newcommand\listfigurename{List of Figures}
\fi
\ifdefined\listtablename
  \renewcommand*\listtablename{List of Tables}
\else
  \newcommand\listtablename{List of Tables}
\fi
\ifdefined\figurename
  \renewcommand*\figurename{Figure}
\else
  \newcommand\figurename{Figure}
\fi
\ifdefined\tablename
  \renewcommand*\tablename{Table}
\else
  \newcommand\tablename{Table}
\fi
}
\@ifpackageloaded{float}{}{\usepackage{float}}
\floatstyle{ruled}
\@ifundefined{c@chapter}{\newfloat{codelisting}{h}{lop}}{\newfloat{codelisting}{h}{lop}[chapter]}
\floatname{codelisting}{Listing}
\newcommand*\listoflistings{\listof{codelisting}{List of Listings}}
\usepackage{amsthm}
\theoremstyle{definition}
\newtheorem{definition}{Definition}[section]
\theoremstyle{remark}
\AtBeginDocument{\renewcommand*{\proofname}{Proof}}
\newtheorem*{remark}{Remark}
\newtheorem*{solution}{Solution}
\newtheorem{refremark}{Remark}[section]
\newtheorem{refsolution}{Solution}[section]
\makeatother
\makeatletter
\makeatother
\makeatletter
\@ifpackageloaded{caption}{}{\usepackage{caption}}
\@ifpackageloaded{subcaption}{}{\usepackage{subcaption}}
\makeatother

\newcommand{\Set}{{S}}
\newcommand{\Monoid}{{M}}
\newcommand{\GroupElement}{{g}}
\newcommand{\Measure}{{\mu}}
\newcommand{\SigmaAlgebra}[1]{{\mathscr{#1}}}
\newcommand{\Folner}[1][\vphantom{}]{{\Phi}_{#1}}
\newcommand{\MonoidElement}{{m}}
\newcommand{\GroupActionPreImage}[2]{{#1}^{-1}.{#2}}
\newcommand{\GroupAction}[2]{{#1}.{#2}}
\NewDocumentCommand{\KoopmanOperator}{O{{\GroupElement}} O{{\vphantom{}}} }{{U}_{{#1}}{#2}}
\newcommand{\GroupOperation}[2]{{#1}\cdot{#2}}
\newcommand{\Group}{{G}}
\newcommand{\GroupIdentity}{{e}}

\usepackage{bookmark}
\IfFileExists{xurl.sty}{\usepackage{xurl}}{} % add URL line breaks if available
\urlstyle{same}
\hypersetup{
  pdftitle={Actions},
  pdfauthor={Kai Prince SFHEA},
  pdflang={en-GB},
  colorlinks=true,
  linkcolor={blue},
  filecolor={Maroon},
  citecolor={Blue},
  urlcolor={Blue},
  pdfcreator={LaTeX via pandoc}}


\title{Actions}
\author{Kai Prince SFHEA}
\date{2025-08-06}
\begin{document}
\maketitle

\renewcommand*\contentsname{Table of contents}
{
\hypersetup{linkcolor=}
\setcounter{tocdepth}{3}
\tableofcontents
}

\begin{definition}[Bekka and Mayer
\citeproc{ref-Bekka2000GroupActionDynamics}{(2000)} Section
2]\protect\hypertarget{def-action}{}\label{def-action}

An \emph{action} of a group, \(\Group\), on a measurable space
\((X,\SigmaAlgebra{B})\) is a measurable mapping
\[\Group\times X\rightarrow X,\ (\GroupElement,x)\mapsto \GroupAction{\GroupElement}{x} \]
with the following properties:

\begin{enumerate}
\def\labelenumi{\arabic{enumi}.}
\tightlist
\item
  Associativity: For all
  \(\GroupElement,\GroupElement'\in \Group,x\in X\), then
  \(\GroupAction{\GroupElement}{(\GroupAction{\GroupElement'}{x})}=\GroupAction{(\GroupOperation{\GroupElement}{\GroupElement'})}{x}\)
\item
  Identity: There exists an identity element \(\GroupIdentity\in\Group\)
  such that \(\GroupAction{\GroupIdentity}{x}=x\) for all \(x\in X\).
\item
  Quasi-Invariance: For any \(B\in\SigmaAlgebra{B}\) and for all
  \(\GroupElement\in\Group\), we have
  \(\Measure(\GroupAction{\GroupElement}{B})=0\) if and only if
  \(\Measure(B)=0\).
\end{enumerate}

The action of \(\Group\) is also ergodic if it satisfies the additional
property:

\begin{enumerate}
\def\labelenumi{\arabic{enumi}.}
\setcounter{enumi}{3}
\tightlist
\item
  If \(B\in\SigmaAlgebra{B}\) and
  \(\Measure(B)=\Measure(\GroupAction{\GroupElement}{B})\) for any
  \(\GroupElement\in\Group\), then \(\Measure(B)=0\) or
  \(\Measure(X\setminus B)=0\).
\end{enumerate}

\end{definition}

\begin{refremark}
We don't require invertability in order to use actions and could instead
use a monoid, \(\Monoid\), defining the pre-image of \(\Monoid\) on
\((X,\SigmaAlgebra{B})\) as a measurable mapping
\[\Monoid\times X\rightarrow \SigmaAlgebra{B},\ (\MonoidElement,x)\mapsto \GroupActionPreImage{\MonoidElement}{x} \]
such that
\[\GroupActionPreImage{\MonoidElement}{x}=\{x'\in X: \GroupAction{\MonoidElement}{x'}=x \}. \]

\label{rem-MonoidAction}

\end{refremark}

\begin{definition}[]\protect\hypertarget{def-topologicalDynamicalSystem}{}\label{def-topologicalDynamicalSystem}

A \emph{topological dynamical system under the action of \(\Group\)},
denoted \((X,\Group)\), is a compact metric space \(X\) that has
continuous surjective maps,
\((\GroupElement,x)\mapsto \GroupAction{\GroupElement}{x}\), for all
\(\GroupElement\in\Group\).

\end{definition}

\begin{definition}[]\protect\hypertarget{def-generic}{}\label{def-generic}

Let \(x\in X\), \(\Folner=(\Folner[N])_{N\in\mathbb{N}}\) be a Følner
sequence in \(\Gamma\) and \(\Measure\) a probability measure on \(X\).
Where \(\delta_x\) is the Dirac mass at \(x\), if
\[\frac{1}{|\Folner[N]|}\sum_{\GroupElement\in\Folner[N]}\delta_{\GroupAction{\GroupElement}{x}}\underset{\text{weakly*}}{\longrightarrow} \Measure \text{ as }N\rightarrow\infty, \]
then we say \emph{\(x\) is generic for \(\Measure\) with respect to
\(\Folner\)} and we denote this with
\(x\in\text{gen}(\Measure,\Folner)\).\footnote{consider tempered
  separately as the FCP construction only depends on sequential
  compactness}

\end{definition}

We are interested in how the action of a group \(\Group\) transforms
functions on \((X,\SigmaAlgebra{B},\Measure)\) so we must identify the
associated definitions within functional analysis.

We define the map
\(\KoopmanOperator:\text{L}^2(X)\rightarrow\text{L}^2(X)\) where
\(f\mapsto f\circ \KoopmanOperator\) as the \emph{Koopman operator}
induced by \(\GroupElement\in\Group\). We find that the Koopman operator
induced by \(\Group\) is a unitary operator and the group homomorphism
\(U:\Group\rightarrow\mathscr{U}(\text{L}^2(X))\) is the \emph{unitary
representation} of \(\Group\) on \(\text{L}^2(X)\), where
\(\mathscr{U}(\text{L}^2(X))\) is the set of all unitary operators on
\(\text{L}^2(X)\).

\phantomsection\label{refs}
\begin{CSLReferences}{1}{0}
\bibitem[\citeproctext]{ref-Bekka2000GroupActionDynamics}
Bekka, M. B. and Mayer, M. (2000). \emph{Ergodic theory and topological
dynamics of group actions on homogeneous spaces}. Cambridge University
Press.
\href{https://doi.org/10.1017/cbo9780511758898}{https://doi.org/10.1017/cbo9780511758898.}

\end{CSLReferences}




\end{document}
