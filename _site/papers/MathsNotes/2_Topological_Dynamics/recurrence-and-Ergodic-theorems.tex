% Options for packages loaded elsewhere
% Options for packages loaded elsewhere
\PassOptionsToPackage{unicode}{hyperref}
\PassOptionsToPackage{hyphens}{url}
\PassOptionsToPackage{dvipsnames,svgnames,x11names}{xcolor}
%
\documentclass[
  british,
]{article}
\usepackage{xcolor}
\usepackage{amsmath,amssymb}
\setcounter{secnumdepth}{5}
\usepackage{iftex}
\ifPDFTeX
  \usepackage[T1]{fontenc}
  \usepackage[utf8]{inputenc}
  \usepackage{textcomp} % provide euro and other symbols
\else % if luatex or xetex
  \usepackage{unicode-math} % this also loads fontspec
  \defaultfontfeatures{Scale=MatchLowercase}
  \defaultfontfeatures[\rmfamily]{Ligatures=TeX,Scale=1}
\fi
\usepackage{lmodern}
\ifPDFTeX\else
  % xetex/luatex font selection
\fi
% Use upquote if available, for straight quotes in verbatim environments
\IfFileExists{upquote.sty}{\usepackage{upquote}}{}
\IfFileExists{microtype.sty}{% use microtype if available
  \usepackage[]{microtype}
  \UseMicrotypeSet[protrusion]{basicmath} % disable protrusion for tt fonts
}{}
\makeatletter
\@ifundefined{KOMAClassName}{% if non-KOMA class
  \IfFileExists{parskip.sty}{%
    \usepackage{parskip}
  }{% else
    \setlength{\parindent}{0pt}
    \setlength{\parskip}{6pt plus 2pt minus 1pt}}
}{% if KOMA class
  \KOMAoptions{parskip=half}}
\makeatother
% Make \paragraph and \subparagraph free-standing
\makeatletter
\ifx\paragraph\undefined\else
  \let\oldparagraph\paragraph
  \renewcommand{\paragraph}{
    \@ifstar
      \xxxParagraphStar
      \xxxParagraphNoStar
  }
  \newcommand{\xxxParagraphStar}[1]{\oldparagraph*{#1}\mbox{}}
  \newcommand{\xxxParagraphNoStar}[1]{\oldparagraph{#1}\mbox{}}
\fi
\ifx\subparagraph\undefined\else
  \let\oldsubparagraph\subparagraph
  \renewcommand{\subparagraph}{
    \@ifstar
      \xxxSubParagraphStar
      \xxxSubParagraphNoStar
  }
  \newcommand{\xxxSubParagraphStar}[1]{\oldsubparagraph*{#1}\mbox{}}
  \newcommand{\xxxSubParagraphNoStar}[1]{\oldsubparagraph{#1}\mbox{}}
\fi
\makeatother


\usepackage{longtable,booktabs,array}
\usepackage{calc} % for calculating minipage widths
% Correct order of tables after \paragraph or \subparagraph
\usepackage{etoolbox}
\makeatletter
\patchcmd\longtable{\par}{\if@noskipsec\mbox{}\fi\par}{}{}
\makeatother
% Allow footnotes in longtable head/foot
\IfFileExists{footnotehyper.sty}{\usepackage{footnotehyper}}{\usepackage{footnote}}
\makesavenoteenv{longtable}
\usepackage{graphicx}
\makeatletter
\newsavebox\pandoc@box
\newcommand*\pandocbounded[1]{% scales image to fit in text height/width
  \sbox\pandoc@box{#1}%
  \Gscale@div\@tempa{\textheight}{\dimexpr\ht\pandoc@box+\dp\pandoc@box\relax}%
  \Gscale@div\@tempb{\linewidth}{\wd\pandoc@box}%
  \ifdim\@tempb\p@<\@tempa\p@\let\@tempa\@tempb\fi% select the smaller of both
  \ifdim\@tempa\p@<\p@\scalebox{\@tempa}{\usebox\pandoc@box}%
  \else\usebox{\pandoc@box}%
  \fi%
}
% Set default figure placement to htbp
\def\fps@figure{htbp}
\makeatother


% definitions for citeproc citations
\NewDocumentCommand\citeproctext{}{}
\NewDocumentCommand\citeproc{mm}{%
  \begingroup\def\citeproctext{#2}\cite{#1}\endgroup}
\makeatletter
 % allow citations to break across lines
 \let\@cite@ofmt\@firstofone
 % avoid brackets around text for \cite:
 \def\@biblabel#1{}
 \def\@cite#1#2{{#1\if@tempswa , #2\fi}}
\makeatother
\newlength{\cslhangindent}
\setlength{\cslhangindent}{1.5em}
\newlength{\csllabelwidth}
\setlength{\csllabelwidth}{3em}
\newenvironment{CSLReferences}[2] % #1 hanging-indent, #2 entry-spacing
 {\begin{list}{}{%
  \setlength{\itemindent}{0pt}
  \setlength{\leftmargin}{0pt}
  \setlength{\parsep}{0pt}
  % turn on hanging indent if param 1 is 1
  \ifodd #1
   \setlength{\leftmargin}{\cslhangindent}
   \setlength{\itemindent}{-1\cslhangindent}
  \fi
  % set entry spacing
  \setlength{\itemsep}{#2\baselineskip}}}
 {\end{list}}
\usepackage{calc}
\newcommand{\CSLBlock}[1]{\hfill\break\parbox[t]{\linewidth}{\strut\ignorespaces#1\strut}}
\newcommand{\CSLLeftMargin}[1]{\parbox[t]{\csllabelwidth}{\strut#1\strut}}
\newcommand{\CSLRightInline}[1]{\parbox[t]{\linewidth - \csllabelwidth}{\strut#1\strut}}
\newcommand{\CSLIndent}[1]{\hspace{\cslhangindent}#1}

\ifLuaTeX
\usepackage[bidi=basic]{babel}
\else
\usepackage[bidi=default]{babel}
\fi
% get rid of language-specific shorthands (see #6817):
\let\LanguageShortHands\languageshorthands
\def\languageshorthands#1{}
\ifLuaTeX
  \usepackage[english]{selnolig} % disable illegal ligatures
\fi


\setlength{\emergencystretch}{3em} % prevent overfull lines

\providecommand{\tightlist}{%
  \setlength{\itemsep}{0pt}\setlength{\parskip}{0pt}}



 


% Get Math Fonts
\usepackage{fontspec}
\usepackage{unicode-math}
\setmathfont{Latin Modern Math}
\setmathfont[range={\mathscr,\mathbfscr,\mathbb}]{XITSMath-Regular}
[    Extension = .otf,
      BoldFont = XITSMath-Bold
]
\makeatletter
\@ifpackageloaded{caption}{}{\usepackage{caption}}
\AtBeginDocument{%
\ifdefined\contentsname
  \renewcommand*\contentsname{Table of contents}
\else
  \newcommand\contentsname{Table of contents}
\fi
\ifdefined\listfigurename
  \renewcommand*\listfigurename{List of Figures}
\else
  \newcommand\listfigurename{List of Figures}
\fi
\ifdefined\listtablename
  \renewcommand*\listtablename{List of Tables}
\else
  \newcommand\listtablename{List of Tables}
\fi
\ifdefined\figurename
  \renewcommand*\figurename{Figure}
\else
  \newcommand\figurename{Figure}
\fi
\ifdefined\tablename
  \renewcommand*\tablename{Table}
\else
  \newcommand\tablename{Table}
\fi
}
\@ifpackageloaded{float}{}{\usepackage{float}}
\floatstyle{ruled}
\@ifundefined{c@chapter}{\newfloat{codelisting}{h}{lop}}{\newfloat{codelisting}{h}{lop}[chapter]}
\floatname{codelisting}{Listing}
\newcommand*\listoflistings{\listof{codelisting}{List of Listings}}
\usepackage{amsthm}
\theoremstyle{plain}
\newtheorem{theorem}{Theorem}[section]
\theoremstyle{plain}
\newtheorem{corollary}{Corollary}[section]
\theoremstyle{remark}
\AtBeginDocument{\renewcommand*{\proofname}{Proof}}
\newtheorem*{remark}{Remark}
\newtheorem*{solution}{Solution}
\newtheorem{refremark}{Remark}[section]
\newtheorem{refsolution}{Solution}[section]
\makeatother
\makeatletter
\makeatother
\makeatletter
\@ifpackageloaded{caption}{}{\usepackage{caption}}
\@ifpackageloaded{subcaption}{}{\usepackage{subcaption}}
\makeatother

\newcommand{\AmenableGroupElement}{{\gamma}}
\newcommand{\Measure}{{\mu}}
\newcommand{\SigmaAlgebra}[1]{{\mathscr{#1}}}
\newcommand{\Folner}[1][\vphantom{}]{{\Phi}_{#1}}
\newcommand{\GroupAction}[2]{{#1}.{#2}}
\newcommand{\AmenableGroup}{{\Gamma}}
\newcommand{\Group}{{G}}
\newcommand{\CountingMeasure}[1][\,\cdot\,]{{\left|#1\right|}}

\usepackage{bookmark}
\IfFileExists{xurl.sty}{\usepackage{xurl}}{} % add URL line breaks if available
\urlstyle{same}
\hypersetup{
  pdftitle={Recurrence and Ergodic Theorems},
  pdfauthor={Kai Prince SFHEA},
  pdflang={en-GB},
  colorlinks=true,
  linkcolor={blue},
  filecolor={Maroon},
  citecolor={Blue},
  urlcolor={Blue},
  pdfcreator={LaTeX via pandoc}}


\title{Recurrence and Ergodic Theorems}
\author{Kai Prince SFHEA}
\date{2025-07-29}
\begin{document}
\maketitle

\renewcommand*\contentsname{Table of contents}
{
\hypersetup{linkcolor=}
\setcounter{tocdepth}{3}
\tableofcontents
}

\begin{theorem}[Bergelson
\citeproc{ref-Bergelson1985Recurrence}{(1985)}, Theorem
1.1]\protect\hypertarget{thm-recurrence}{}\label{thm-recurrence}

Let \((X,\SigmaAlgebra{B},\Measure)\) be a probability space and suppose
that \(B_n\in\SigmaAlgebra{B}\) such that \(\Measure(B_n)=b>0\) for all
\(n\in\mathbb{N}\).

Then there exists a positively dense index set \(I\subset\mathbb{N}\)
such that, for any finite subset \(F\subseteq I\), we have
\[\Measure\left(\bigcap_{i\in F}B_i \right)>0. \]

\end{theorem}

\begin{theorem}[cf. Lindenstrauss
\citeproc{ref-Lindenstrauss2001PETAmenable}{(2001)}, Theorem
1.2]\protect\hypertarget{thm-GeneralisedPET}{}\label{thm-GeneralisedPET}

Let \(\AmenableGroup\) be a discrete amenable group acting on a measure
space \((X,\SigmaAlgebra{B},\Measure)\) by measure preserving
transformation and let \(\Folner=(\Folner[N])_{N\in\mathbb{N}}\) be a
tempered Følner sequence.

Then, for any \(f\in\text{L}^1(\Measure)\), there is a
\(\AmenableGroup\)-invariant \(\bar{f}\in\text{L}^1(\Measure)\) such
that
\[\lim_{N\rightarrow\infty}\frac{1}{\CountingMeasure{\Folner[N]}}\sum_{\AmenableGroupElement\in\Folner[N]}f(\GroupAction{\AmenableGroupElement}{x})=\bar{f}(x) \]
for \(\Measure\)-almost every \(x\in X\). In particular, if the
\(\AmenableGroup\) action is ergodic, then
\[\lim_{N\rightarrow\infty}\frac{1}{\CountingMeasure{\Folner[N]}}\sum_{\AmenableGroupElement\in\Folner[N]}f(\GroupAction{\AmenableGroupElement}{x})=\int f(x)\ d\Measure(x) \]
for \(\Measure\) almost every \(x\).

\end{theorem}

\begin{corollary}[cf. Host \citeproc{ref-host2019short}{(2019)},
Corollary
8]\protect\hypertarget{cor-temperedGeneric}{}\label{cor-temperedGeneric}

Let \((X,\AmenableGroup)\) be a topological dynamical system where
\(\AmenableGroup\) is an amenable group, \(\Measure\) an ergodic measure
on \(X\) and \(\Folner\) a tempered Følner sequence. Then
\(\Measure\)-almost every \(x\in X\) is generic for \(\Measure\) along
\(\Folner\).

\end{corollary}

\phantomsection\label{refs}
\begin{CSLReferences}{1}{0}
\bibitem[\citeproctext]{ref-Bergelson1985Recurrence}
Bergelson, V. (1985). \textquotesingle Sets of recurrence of zm-actions
and properties of sets of differences in zm\textquotesingle,
\emph{\emph{Journal of the London Mathematical Society},} s2-31 (2), pp.
295--304.
\href{https://doi.org/10.1112/jlms/s2-31.2.295}{https://doi.org/10.1112/jlms/s2-31.2.295.}

\bibitem[\citeproctext]{ref-Lindenstrauss2001PETAmenable}
Lindenstrauss, E. (2001). \textquotesingle Pointwise theorems for
amenable groups\textquotesingle, \emph{\emph{Inventiones mathematicae},}
146 (2), pp. 259--295.
\href{https://doi.org/10.1007/s002220100162}{https://doi.org/10.1007/s002220100162.}

\bibitem[\citeproctext]{ref-host2019short}
Host, B. (2019). \textquotesingle{}\emph{A short proof of a conjecture
of erdös proved by moreira, richter and robertson}\textquotesingle,
Available at: \url{https://arxiv.org/abs/1904.09952}.

\end{CSLReferences}




\end{document}
