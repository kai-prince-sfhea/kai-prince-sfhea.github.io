% Options for packages loaded elsewhere
% Options for packages loaded elsewhere
\PassOptionsToPackage{unicode}{hyperref}
\PassOptionsToPackage{hyphens}{url}
\PassOptionsToPackage{dvipsnames,svgnames,x11names}{xcolor}
%
\documentclass[
  british,
]{article}
\usepackage{xcolor}
\usepackage{amsmath,amssymb}
\setcounter{secnumdepth}{5}
\usepackage{iftex}
\ifPDFTeX
  \usepackage[T1]{fontenc}
  \usepackage[utf8]{inputenc}
  \usepackage{textcomp} % provide euro and other symbols
\else % if luatex or xetex
  \usepackage{unicode-math} % this also loads fontspec
  \defaultfontfeatures{Scale=MatchLowercase}
  \defaultfontfeatures[\rmfamily]{Ligatures=TeX,Scale=1}
\fi
\usepackage{lmodern}
\ifPDFTeX\else
  % xetex/luatex font selection
\fi
% Use upquote if available, for straight quotes in verbatim environments
\IfFileExists{upquote.sty}{\usepackage{upquote}}{}
\IfFileExists{microtype.sty}{% use microtype if available
  \usepackage[]{microtype}
  \UseMicrotypeSet[protrusion]{basicmath} % disable protrusion for tt fonts
}{}
\makeatletter
\@ifundefined{KOMAClassName}{% if non-KOMA class
  \IfFileExists{parskip.sty}{%
    \usepackage{parskip}
  }{% else
    \setlength{\parindent}{0pt}
    \setlength{\parskip}{6pt plus 2pt minus 1pt}}
}{% if KOMA class
  \KOMAoptions{parskip=half}}
\makeatother
% Make \paragraph and \subparagraph free-standing
\makeatletter
\ifx\paragraph\undefined\else
  \let\oldparagraph\paragraph
  \renewcommand{\paragraph}{
    \@ifstar
      \xxxParagraphStar
      \xxxParagraphNoStar
  }
  \newcommand{\xxxParagraphStar}[1]{\oldparagraph*{#1}\mbox{}}
  \newcommand{\xxxParagraphNoStar}[1]{\oldparagraph{#1}\mbox{}}
\fi
\ifx\subparagraph\undefined\else
  \let\oldsubparagraph\subparagraph
  \renewcommand{\subparagraph}{
    \@ifstar
      \xxxSubParagraphStar
      \xxxSubParagraphNoStar
  }
  \newcommand{\xxxSubParagraphStar}[1]{\oldsubparagraph*{#1}\mbox{}}
  \newcommand{\xxxSubParagraphNoStar}[1]{\oldsubparagraph{#1}\mbox{}}
\fi
\makeatother


\usepackage{longtable,booktabs,array}
\usepackage{calc} % for calculating minipage widths
% Correct order of tables after \paragraph or \subparagraph
\usepackage{etoolbox}
\makeatletter
\patchcmd\longtable{\par}{\if@noskipsec\mbox{}\fi\par}{}{}
\makeatother
% Allow footnotes in longtable head/foot
\IfFileExists{footnotehyper.sty}{\usepackage{footnotehyper}}{\usepackage{footnote}}
\makesavenoteenv{longtable}
\usepackage{graphicx}
\makeatletter
\newsavebox\pandoc@box
\newcommand*\pandocbounded[1]{% scales image to fit in text height/width
  \sbox\pandoc@box{#1}%
  \Gscale@div\@tempa{\textheight}{\dimexpr\ht\pandoc@box+\dp\pandoc@box\relax}%
  \Gscale@div\@tempb{\linewidth}{\wd\pandoc@box}%
  \ifdim\@tempb\p@<\@tempa\p@\let\@tempa\@tempb\fi% select the smaller of both
  \ifdim\@tempa\p@<\p@\scalebox{\@tempa}{\usebox\pandoc@box}%
  \else\usebox{\pandoc@box}%
  \fi%
}
% Set default figure placement to htbp
\def\fps@figure{htbp}
\makeatother


% definitions for citeproc citations
\NewDocumentCommand\citeproctext{}{}
\NewDocumentCommand\citeproc{mm}{%
  \begingroup\def\citeproctext{#2}\cite{#1}\endgroup}
\makeatletter
 % allow citations to break across lines
 \let\@cite@ofmt\@firstofone
 % avoid brackets around text for \cite:
 \def\@biblabel#1{}
 \def\@cite#1#2{{#1\if@tempswa , #2\fi}}
\makeatother
\newlength{\cslhangindent}
\setlength{\cslhangindent}{1.5em}
\newlength{\csllabelwidth}
\setlength{\csllabelwidth}{3em}
\newenvironment{CSLReferences}[2] % #1 hanging-indent, #2 entry-spacing
 {\begin{list}{}{%
  \setlength{\itemindent}{0pt}
  \setlength{\leftmargin}{0pt}
  \setlength{\parsep}{0pt}
  % turn on hanging indent if param 1 is 1
  \ifodd #1
   \setlength{\leftmargin}{\cslhangindent}
   \setlength{\itemindent}{-1\cslhangindent}
  \fi
  % set entry spacing
  \setlength{\itemsep}{#2\baselineskip}}}
 {\end{list}}
\usepackage{calc}
\newcommand{\CSLBlock}[1]{\hfill\break\parbox[t]{\linewidth}{\strut\ignorespaces#1\strut}}
\newcommand{\CSLLeftMargin}[1]{\parbox[t]{\csllabelwidth}{\strut#1\strut}}
\newcommand{\CSLRightInline}[1]{\parbox[t]{\linewidth - \csllabelwidth}{\strut#1\strut}}
\newcommand{\CSLIndent}[1]{\hspace{\cslhangindent}#1}

\ifLuaTeX
\usepackage[bidi=basic]{babel}
\else
\usepackage[bidi=default]{babel}
\fi
% get rid of language-specific shorthands (see #6817):
\let\LanguageShortHands\languageshorthands
\def\languageshorthands#1{}
\ifLuaTeX
  \usepackage[english]{selnolig} % disable illegal ligatures
\fi


\setlength{\emergencystretch}{3em} % prevent overfull lines

\providecommand{\tightlist}{%
  \setlength{\itemsep}{0pt}\setlength{\parskip}{0pt}}



 


% Get Math Fonts
\usepackage{fontspec}
\usepackage{unicode-math}
\setmathfont{Latin Modern Math}
\setmathfont[range={\mathscr,\mathbfscr,\mathbb}]{XITSMath-Regular}
[    Extension = .otf,
      BoldFont = XITSMath-Bold
]
\makeatletter
\@ifpackageloaded{caption}{}{\usepackage{caption}}
\AtBeginDocument{%
\ifdefined\contentsname
  \renewcommand*\contentsname{Table of contents}
\else
  \newcommand\contentsname{Table of contents}
\fi
\ifdefined\listfigurename
  \renewcommand*\listfigurename{List of Figures}
\else
  \newcommand\listfigurename{List of Figures}
\fi
\ifdefined\listtablename
  \renewcommand*\listtablename{List of Tables}
\else
  \newcommand\listtablename{List of Tables}
\fi
\ifdefined\figurename
  \renewcommand*\figurename{Figure}
\else
  \newcommand\figurename{Figure}
\fi
\ifdefined\tablename
  \renewcommand*\tablename{Table}
\else
  \newcommand\tablename{Table}
\fi
}
\@ifpackageloaded{float}{}{\usepackage{float}}
\floatstyle{ruled}
\@ifundefined{c@chapter}{\newfloat{codelisting}{h}{lop}}{\newfloat{codelisting}{h}{lop}[chapter]}
\floatname{codelisting}{Listing}
\newcommand*\listoflistings{\listof{codelisting}{List of Listings}}
\usepackage{amsthm}
\theoremstyle{plain}
\newtheorem{theorem}{Theorem}[section]
\theoremstyle{definition}
\newtheorem{definition}{Definition}[section]
\theoremstyle{remark}
\AtBeginDocument{\renewcommand*{\proofname}{Proof}}
\newtheorem*{remark}{Remark}
\newtheorem*{solution}{Solution}
\newtheorem{refremark}{Remark}[section]
\newtheorem{refsolution}{Solution}[section]
\makeatother
\makeatletter
\makeatother
\makeatletter
\@ifpackageloaded{caption}{}{\usepackage{caption}}
\@ifpackageloaded{subcaption}{}{\usepackage{subcaption}}
\makeatother
\newcommand{\AmenableGroupElement}{{\gamma}}
\newcommand{\AmenableGroup}{{\Gamma}}
\newcommand{\Folner}[1][\vphantom{}]{{\Phi}_{#1}}
\newcommand{\GroupAction}[2]{{#1}.{#2}}
\newcommand{\GroupElement}{{g}}
\newcommand{\GroupIdentity}{{e}}
\newcommand{\GroupOperation}[2]{{#1}\cdot{#2}}
\newcommand{\Group}{{G}}
\newcommand{\Identity}{{e}}
\newcommand{\Inverse}[1]{{#1}^{-1}}
\newcommand{\N}{\mathbb{N}}
\newcommand{\Operation}[2]{{#1}\cdot{#2}}
\newcommand{\Set}{{S}}
\usepackage{bookmark}
\IfFileExists{xurl.sty}{\usepackage{xurl}}{} % add URL line breaks if available
\urlstyle{same}
\hypersetup{
  pdfauthor={Kai Prince},
  pdflang={en-GB},
  colorlinks=true,
  linkcolor={blue},
  filecolor={Maroon},
  citecolor={Blue},
  urlcolor={Blue},
  pdfcreator={LaTeX via pandoc}}


\author{Kai Prince}
\date{2025-07-23}
\begin{document}

\renewcommand*\contentsname{Table of contents}
{
\hypersetup{linkcolor=}
\setcounter{tocdepth}{3}
\tableofcontents
}

\section*{Abstract}\label{abstract}
\addcontentsline{toc}{section}{Abstract}

We follow steps provided by the paper published by
\citeproc{ref-host2019short}{(Host, 2019)}, `A Short Proof of a
Conjecture of Erdős Proved by Moreira, Richter and Robertson', as well
as using the results provided by \citeproc{ref-kra2022infinite}{(Kra et
al., 2022)}, to generalise the proof of Erdős's conjecture for amenable
groups. The main result that we aim to prove is: `every positively dense
subset of an amenable group contains the group sumset of \(k\) infinite
sets for every natural number \(k\)'.

\section{Introduction}\label{introduction}

\begin{theorem}[cf. \citeproc{ref-host2019short}{(Host, 2019)}, Theorem
1]\protect\hypertarget{thm-GenErdosConjComb}{}\label{thm-GenErdosConjComb}

Let \((\Group,\cdot)\) be an amenable group. If \(A\subseteq G\) has
positive density, then there exists infinite subsets \(B\) and \(C\) of
\(G\) such that \(B\cdot C\subset A\).

\end{theorem}

\begin{theorem}[cf. \citeproc{ref-host2019short}{(Host, 2019)},
Proposition
2]\protect\hypertarget{thm-GenErdosConjRel}{}\label{thm-GenErdosConjRel}

There exists a set of positive density not containing any sumset of
positive density and an infinite set.

\end{theorem}

\begin{theorem}[cf. \citeproc{ref-kra2022infinite}{(Kra et al., 2022)},
Theorem
1.1]\protect\hypertarget{thm-GenNErdosConjComb}{}\label{thm-GenNErdosConjComb}

Let \((G,\cdot)\) be an amenable group. If \(A\subseteq G\) has positive
density then, for every \(k\in\mathbb{N}\), there are infinite subsets
\(B_1,...,B_k\subset G\) such that \(B_1\cdot\cdots\cdot B_k\subset A\).

\end{theorem}

\section{Preliminaries}\label{preliminaries}

We will use \(\mathbb{N}=\{1,2,3,...\}\) and \(\Identity\) to denote the
identity of the group \(\Group\).

\subsection{Amenable Groups and
Actions}\label{amenable-groups-and-actions}

Throughout this paper, unless otherwise specified, we will let
\((G,\cdot)\) be a second countable discrete group. This also means that
the Haar measure of \(G\) is the counting measure.

\begin{definition}[]\protect\hypertarget{def-FolnerSequenceDef}{}\label{def-FolnerSequenceDef}

{[}Embed definition from folner.qmd{]}

\end{definition}

For simplicity of this paper, we will use the alternative and equivalent
definition that a group, \(\AmenableGroup\), is \emph{amenable} (and
second countable) if and only if it has a Følner sequence.

\begin{definition}[]\protect\hypertarget{def-TemperedSequenceDef}{}\label{def-TemperedSequenceDef}

{[}Embed definition from folner.qmd{]}

\end{definition}

\begin{definition}[]\protect\hypertarget{def-DensityDef}{}\label{def-DensityDef}

{[}Embed definition from density.qmd{]}

\end{definition}

\subsection{Topological Dynamics of Group
Actions}\label{topological-dynamics-of-group-actions}

\begin{definition}[]\protect\hypertarget{def-ActionDef}{}\label{def-ActionDef}

{[}Embed definition from actions.qmd{]}

\end{definition}

A \emph{topological dynamical system under the action of \(\Group\)},
denoted \((X,\Group)\), is a compact metric space \(X\) that has
continuous surjective maps,
\((\GroupElement,x)\mapsto \GroupAction{\GroupElement}{x}\), for all
\(\GroupElement\in\Group\).

Let \(x\in X\), \(\Folner=(\Folner[N])_{N\in\mathbb{N}}\) be a Følner
sequence in \(\Gamma\) and \(\mu\) a probability measure on \(X\). Where
\(\delta_x\) is the Dirac mass at \(x\), if
\[\frac{1}{|\Folner[N]|}\sum_{\GroupElement\in\Folner[N]}\delta_{\GroupAction{\GroupElement}{x}}\underset{\text{weakly*}}{\longrightarrow} \mu \text{ as }N\rightarrow\infty, \]
then we say \emph{\(x\) is generic for \(\mu\) with respect to
\(\Folner\)} and we denote this with
\(x\in\text{gen}(\mu,\Phi)\).\footnote{consider tempered separately as
  the FCP construction only depends on sequential compactness}

\subsection{Recurrence Results \& Ergodic
Theorems}\label{recurrence-results-ergodic-theorems}

\subsection{Erdős Cubes \& Cubic
Measures}\label{erdux151s-cubes-cubic-measures}

\subsection{Factor Maps}\label{factor-maps}

\subsection{Key Dynamical Results}\label{key-dynamical-results}

\section{Proof of Theorem (ref)}\label{proof-of-theorem-ref}

\subsection{Furstenberg's Correspondence
Princple}\label{furstenbergs-correspondence-princple}

\subsection{Kronecker Factor}\label{kronecker-factor}

\subsection{\texorpdfstring{Choosing a point
\(x_1\)}{Choosing a point x\_1}}\label{choosing-a-point-x_1}

\subsection{\texorpdfstring{The joining
\(\nu\)}{The joining \textbackslash nu}}\label{the-joining-nu}

\subsection{Proof Conclusion}\label{proof-conclusion}

\section{Proof of Corollary (ref)}\label{proof-of-corollary-ref}

\section{Proof of Theorem (ref)}\label{proof-of-theorem-ref-1}

\section*{Discussion}\label{discussion}
\addcontentsline{toc}{section}{Discussion}

\phantomsection\label{refs}
\begin{CSLReferences}{1}{0}
\bibitem[\citeproctext]{ref-host2019short}
Host, B. (2019). \textquotesingle{}\emph{A short proof of a conjecture
of erdös proved by moreira, richter and robertson}\textquotesingle,
Available at: \url{https://arxiv.org/abs/1904.09952}.

\bibitem[\citeproctext]{ref-kra2022infinite}
Kra, B., et al. (2022). \textquotesingle{}\emph{Infinite sumsets in sets
with positive density}\textquotesingle, Available at:
\url{https://arxiv.org/abs/2206.01786}.

\end{CSLReferences}




\end{document}
